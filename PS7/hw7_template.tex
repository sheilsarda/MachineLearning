\documentclass[english]{article}
\usepackage{comment}
\usepackage[letterpaper]{geometry}
\usepackage{tikz}
\usepackage{amsmath}
\usepackage{amsfonts}
\usepackage{amssymb}
\usepackage{float}
\usepackage{graphicx}
\geometry{verbose,tmargin=1in,bmargin=1in,lmargin=1in,rmargin=1in}


\title{CIS 520, Machine Learning, Fall 2020 \\ Homework 7\\
Due: Monday, November 23rd, 11:59pm \\
Submit to Gradescope}
\date{}
\author{Yuezhong Chen, Sheil Sarda}


\begin{document}
\maketitle

\section*{Problem 1}
\begin{enumerate}
    \item 
    \begin{enumerate}
        \item It is appropriate as causes of precipitation include many other factors besides previous day precipitation.
        \item Not appropriate as there is nothing hidden.
        \item Appropriate as factors other than preferences may affect the pleasure from a specific movie.
        \item Appropriate, as many factors affect market prices besides latest prices.
    \end{enumerate}
    \item
    \begin{enumerate}
        \item False, as Markov models depend only on the last state.
        \item True, as it give more parameters and more freedom to fit the training data.
    \end{enumerate}
    \item This probability may certainly depend on the time of the day which is not a random variable, so a vanilla HMM with a constant transition matrix would not work well.
    \item One can take a grid of times in a 24 hours period and us an independent transition matrix for each such time.
\end{enumerate}

\section*{Problem 2}
\begin{enumerate}
    \item \begin{enumerate}
        \item $Z_2 = ReLU(UX_2+WZ_1); O_2 = softmax(VZ_2)$
        \item Because $Z_2 = ReLU(UX_2+WZ_1)$, we firstly calculate the following:
        \[
         UX_2+WZ_1 = 
         \begin{pmatrix}
                 5.5 & 2.3 & 2.0 \\
                 3.2 & 7.1 & 0.5
         \end{pmatrix}
         \begin{pmatrix}
                 0 \\
                 0 \\
                 1
         \end{pmatrix}
         +
         \begin{pmatrix}
                 0.5 & 0.2 \\
                 2.0 & 0.9
         \end{pmatrix}
         \begin{pmatrix}
                 0.5 \\
                 0.5
         \end{pmatrix}
         =
         \begin{pmatrix}
                 2.0 \\
                 0.5
         \end{pmatrix}
         +
         \begin{pmatrix}
                 0.35 \\
                 1.45
         \end{pmatrix}
        \]
        Thus, $Z_2 = 
        \begin{pmatrix}
                 2.35 \\
                 1.95
         \end{pmatrix}$. And because $O_2 = softmax(VZ_2)$
         \[
         VZ_2 =
         \begin{pmatrix}
                 0.4 & 1.3 \\
                 0.6 & 0.9 \\
                 1.6 & 0.3
         \end{pmatrix}
         \begin{pmatrix}
                 2.35 \\
                 1.95
         \end{pmatrix}=
         \begin{pmatrix}
                 3.475 \\
                 3.165 \\
                 4.345
         \end{pmatrix}
         \]
         Thus, $O_2= softmax
         \begin{pmatrix}
                 3.475 \\
                 3.165 \\
                 4.345
         \end{pmatrix} \approx 
         \begin{pmatrix}
                 0.243 \\
                 0.178 \\
                 0.579
         \end{pmatrix}$.
         
        \item It is watching TV with probability 0.5793.
    \end{enumerate}
    \item \begin{enumerate}
        \item $Z_3 = ReLU(UX_3+WZ_2) = ReLU(UO_2+WZ_2); O_3 = softmax(VZ_3)$
        \item Following the same strategies above, we have:
        \[
        Z_3 \approx
        \begin{pmatrix}
                 3.25 \\
                 1.95
         \end{pmatrix}, 
         O_3 \approx
        \begin{pmatrix}
                 0.9047 \\
                 0.0658 \\
                 0.0295
         \end{pmatrix}
        \]
        \item It is singing with probability 0.9047.
    \end{enumerate}
\end{enumerate}

\section*{Problem 3}
\begin{enumerate}
    \item $p\left(X_{1}, X_{2}, X_{3}, X_{4}, X_{5}, X_{6}\right)= p(X_1)p(X_2)p(X_3 | X_1)p(X_4 | X_1, X_2)p(X_5 | X_3, X_4)p(X_6 | X_4)$
    
    \item $p\left(X_{1}, X_{2}, X_{3}, X_{4}, X_{5}, X_{6}\right)=p\left(X_{1}\right) p\left(X_{2}\right) p\left(X_{3}\right) p\left(X_{4}\right) p\left(X_{5} \mid X_{3}\right) p\left(X_{6} \mid X_{3}\right)$
    
    No, it is not included since this distribution has a restriction on $X_6 | X_3$ whereas the original network has a restriction on $X_6 | X_4$.
    \item If the edge from $X_3$ to $X_5$ is removed, will the class of joint probability distributions that can be represented by the resulting Bayesian network be smaller or larger than that associated with the original network?
    \newline
    The class of joint probability distributions would be larger if the edge were removed, because we are removing the restriction on $X_5 | X_3$.
    \item 
    \begin{enumerate}
        \item $p\left(X_{1}, X_{2}\right)=p\left(X_{1}\right) p\left(X_{2}\right)$. True, are not connected by a path.
        \item $p\left(X_{3}, X_{6} \mid X_{4}\right)=p\left(X_{3} \mid X_{4}\right) p\left(X_{6} \mid X_{4}\right)$. True, $X_3$ and $X_6$ are not connected.
        \item $p\left(X_{1}, X_{2} \mid X_{6}\right)=p\left(X_{1} \mid X_{6}\right) p\left(X_{2} \mid X_{6}\right)$. True, $X_1$ and $X_2$ are not connected.
        \item $p\left(X_{2}, X_{5} \mid X_{4}\right)=p\left(X_{2} \mid X_{4}\right) p\left(X_{5} \mid X_{4}\right)$. False, $X_2$ and $X_5$ are connected through $X_4$.
    \end{enumerate}
\end{enumerate}
\section*{Problem 4}
\begin{enumerate}
    \item 4.1 Nothing to Report. Please skip.
    \item Do you add the link(yes/no)?
    \\ Steps: 
    
    $\operatorname{P}(B)$ = 0.706 \newline
    $\operatorname{P}(B|A)$ = 0.727 \newline
    $\operatorname{P}(B|A^C$) = 0.667
    
    Since the difference between conditional probabilities and the probability of the event is less than 0.05, A and B are independent. Do not add the link.
    
    \item Do you add the link(yes/no)?
	\\ Steps: 
    
    $\operatorname{P}(C) = 0.588$ \newline
    $\operatorname{P}(C|A) = 0.545$ \newline
    $\operatorname{P}(C|A^C) = 0.667$
    
    Since the difference between conditional probabilities and the probability of the event is greater than 0.05, add the link.
    
    \item Do you add the link(yes/no)?
	\\ Steps: 
    
    $\operatorname{P}(C | B) = 0.583$ \newline
    $\operatorname{P}(C | B^C) = 0.600$ 
        
    Since the difference between conditional probabilities and the probability of the event is less than 0.05, do not add the link.
    
    \item Do you add the link(yes/no)?
	\\ Steps: 
    
    $\operatorname{P}(D) = 0.412$ \newline
    $\operatorname{P}(D | A) = 0.454$ \newline
    $\operatorname{P}(D | A^C) = 0.333$ 
        
    Since the difference between conditional probabilities and the probability of the event is greater than 0.05, add the link.
    
    \item Do you add the link(yes/no)?
	\\ Steps: 
    
    $\operatorname{P}(D | B) = 0.454 $\newline
    $\operatorname{P}(D | B^C) = 0.600 $
        
    Since the difference between conditional probabilities and the probability of the event is greater than 0.05, add the link.
    
    \item Do you add the link(yes/no)?
	\\ Steps: 
    
    $\operatorname{P}(D | C) = 0.700$ 
        
    Since the difference between conditional probabilities and the probability of the event is greater than 0.05, add the link.
    
    \item 
    % This code uses the tikz package
    \begin{tikzpicture}
    \draw[step=9cm,white,very thin] (-2,-2) grid (3,3);
   
    \node (v0) at (0.458,-1) {A};
    \node (v1) at (2,-1) {B};
    \node (v2) at (1,0) {C};
    \node (v3) at (2,0) {D};
    \draw [->] (v0) edge (v2);
    \draw [->] (v0) edge (v3);
    \draw [->] (v1) edge (v3);
    \draw [->] (v2) edge (v3);
    \end{tikzpicture}
\end{enumerate}
\end{document}